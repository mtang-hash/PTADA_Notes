\documentclass[12pt,a4paper]{article}
\usepackage[english]{babel}
\usepackage[T1]{fontenc}
\usepackage[utf8]{inputenc}
% \usepackage{lmodern}
\usepackage{lmodern}
\usepackage[left=28mm,top=28mm,right=28mm,bottom=28mm] {geometry}
\usepackage{amsfonts}

\usepackage{mathrsfs}
\usepackage[intlimits]{amsmath}
\usepackage{mathtools}

\usepackage{mathabx}
\usepackage{bbm}
\usepackage{stmaryrd}
\usepackage{relsize}
\usepackage{etoolbox}
\usepackage{lscape}
\usepackage{afterpage}
\usepackage{makecell}
\usepackage[shortlabels]{enumitem}
\usepackage{amssymb}
\usepackage{stmaryrd}
\usepackage{cancel}
\usepackage{mdframed}
\usepackage{framed}{}
\usepackage{tablefootnote} 
\usepackage{listings}
\usepackage{amsthm}
%\usepackage[dvipsnames]{xcolor}
\usepackage[x11names,dvipsnames]{xcolor}
\colorlet{myGray}{gray!3!white}
\usepackage{etoolbox}
\usepackage[all]{xy}
\usepackage{tikz}
%\usepackage{subfigure}
%\usepackage{bbold}
\usetikzlibrary{calc}
\usetikzlibrary{shapes.geometric}
\usetikzlibrary{arrows.meta}
\usetikzlibrary{decorations.pathreplacing}
\usetikzlibrary{fadings}
\usepackage{thmtools}
\usepackage{stmaryrd}
\usepackage[ruled,vlined]{algorithm2e}
%\usepackage{stix}
%\let\Hermaphrodite\relax
\usepackage[font=small]{caption}
\usepackage{subcaption}
\usepackage[final]{pdfpages}
\usepackage{hhline}
\let\Sun\relax
\let\Moon\relax
\let\Mercury\relax
\let\Venus\relax
\let\Earth\relax
\let\Mars\relax
\let\Jupiter\relax
\let\Saturn\relax
\let\Uranus\relax
\let\Neptune\relax
\let\Pluto\relax
\let\Gemini\relax
\let\Leo\relax
\let\Libra\relax
\let\Scorpio\relax
\let\Aries\relax
\let\Taurus\relax
\usepackage{marvosym}
\usepackage[
   pdfpagelabels=true,
   pdftitle={Probabilistic Techniques and Algorithms in Data Analysis},
   pdfauthor={Min Tang},
   hidelinks
 ]{hyperref}
\usepackage{esint}
\usepackage{bookmark}
\usepackage[usenames,dvipsnames]{pstricks}
\usepackage{epsfig}
\usepackage{pst-grad} % For gradients
\usepackage{pst-plot} % For axes
\usepackage[space]{grffile} % For spaces in paths
\usepackage{etoolbox} % For spaces in paths
\usepackage{listings}
\usepackage[style=numeric,sorting=none,backend=bibtex]{biblatex}
\usepackage{dsfont}
\addbibresource{literature.bib}
\lstset{
  frame=none,
  xleftmargin=0pt,
  %stepnumber=1,
  %numbers=left,
  %numbersep=5pt,
  backgroundcolor = \color{myGray},
  numberstyle=\ttfamily\tiny\color[gray]{0.3},
  belowcaptionskip=\bigskipamount,
  captionpos=b,
  escapeinside={*'}{'*},
  language=haskell,
  tabsize=2,
  emphstyle={\bf},
  commentstyle=\it,
  stringstyle=\mdseries\rmfamily,
  showspaces=false,
  keywordstyle=\bfseries\rmfamily,
  columns=flexible,
  basicstyle=\small\sffamily,
  showstringspaces=false,
  morecomment=[l]\%,
}
\makeatletter % For spaces in paths
\patchcmd\Gread@eps{\@inputcheck#1 }{\@inputcheck"#1"\relax}{}{}
\makeatother

\usetikzlibrary{cd}
\usetikzlibrary{calc}
\def\checkmark{\tikz\fill[scale=0.4](0,.35) -- (.25,0) -- (1,.7) -- (.25,.15) -- cycle;} 
\theoremstyle{definition}
\newtheorem{definition}{Definition}
\newtheorem{problem}[definition]{Problem}
\theoremstyle{theorem}
\newtheorem{theorem}[definition]{Theorem}
\newtheorem*{theorem*}{Theorem}
\newtheorem{prop}[definition]{Proposition}
\newtheorem{lemma}[definition]{Lemma}
\newtheorem{corollary}[definition]{Corollary}
\newtheorem{example}[definition]{Example}
\newtheorem*{example*}{Example}
\theoremstyle{definition}
\newtheorem{remark}[definition]{Remark}
\newtheorem*{remark*}{Remark}
\AfterEndEnvironment{lemma}{\noindent\ignorespaces}
\AfterEndEnvironment{definition}{\noindent\ignorespaces}
\AfterEndEnvironment{example}{\noindent\ignorespaces}
\AfterEndEnvironment{theorem}{\noindent\ignorespaces}
% \AfterEndEnvironment{satz}{\noindent\ignorespaces}
% \AfterEndEnvironment{korollar}{\noindent\ignorespaces}
\AfterEndEnvironment{remark}{\noindent\ignorespaces}
\AfterEndEnvironment{remark*}{\noindent\ignorespaces}
% \AfterEndEnvironment{remark'}{\noindent\ignorespaces}
% \AfterEndEnvironment{proposition}{\noindent\ignorespaces}
% \AfterEndEnvironment{proof}{\noindent\ignorespaces}
\let\existstemp\exists
\let\foralltemp\forall
\newcommand{\tikzmark}[1]{\tikz[overlay,remember picture] \node (#1) {};}
\newcommand{\vsubset}{\rotatebox[origin=c]{90}{$\subset$}}
\newcommand{\vphi}{\phi}
\newcommand{\ol}{\overline}
%Differentiation
\newcommand{\D}{\, \mathrm{d} }
%Bold Symbols
\newcommand{\E}{\mathbb{E}}

\newcommand{\R}{\mathbb{R}}
\newcommand{\C}{\mathbb{C}}
\newcommand{\N}{\mathbb{N}}
\newcommand{\Q}{\mathbb{Q}}
%Calligraphic Symbols
\newcommand{\ZZ}{\mathcal{Z}}
\newcommand{\II}{\mathcal{I}}
\newcommand{\FF}{\mathcal{F}}
\newcommand{\QQ}{\mathcal{Q}}
\newcommand{\EE}{\mathcal{E}}
\newcommand{\PP}{\mathcal{P}}
\newcommand{\TT}{\mathcal{T}}
\newcommand{\MM}{\mathcal{M}}
\newcommand{\HH}{\mathscr{H}}
\newcommand{\RR}{\mathscr{R}}
\newcommand{\BB}{\mathscr{B}}
\newcommand{\CC}{\mathscr{C}}
\newcommand{\DD}{\mathscr{D}}
\newcommand{\LL}{\mathscr{L}}
\newcommand{\NN}{\mathscr{N}}
\renewcommand{\AA}{\mathscr{A}}
% Operator names
\newcommand{\id}{\operatorname{id}}
\newcommand{\Hom}{\operatorname{Hom}}
\newcommand{\del}{\partial}
\newcommand{\GL}{\operatorname{GL}}
\newcommand{\vol}{\operatorname{vol}}
\newcommand{\Var}{\operatorname{Var}} 
\newcommand{\Cov}{\operatorname{Cov}}
\newcommand{\End}{\operatorname{End}}
\newcommand{\SL}{\operatorname{SL}}
\newcommand{\Aff}{\operatorname{Aff}}
\newcommand{\Isom}{\operatorname{Isom}}
\newcommand{\Trans}{\operatorname{Trans}}
\newcommand{\Bild}{\begin{tiny}(Bild hier)\end{tiny}}
\newcommand{\Ueb}{\begin{tiny}\textbf{(Ü)}\end{tiny}}
\newcommand{\te}{\text}
\newcommand{\comment}[1]{}
\newcommand{\drawaline}{\rule{\textwidth}{0.2pt}}
%\renewcommand{\def}{\definition}
\renewcommand{\ker}{\operatorname{ker}}
\renewcommand*{\forall}{\foralltemp\mkern2mu}
\renewcommand{\emptyset}{\varnothing}
\renewcommand{\Re}{\operatorname{Re}}
\renewcommand{\O}{\operatorname{O}}
\renewcommand{\Im}{\operatorname{im}}
\renewcommand{\qedsymbol}{$\blacksquare$}
\renewcommand{\phi}{\varphi}
\makeatletter 
\AfterEndEnvironment{mdframed}{%
 \tfn@tablefootnoteprintout% 
 \gdef\tfn@fnt{0}% 
}
\numberwithin{equation}{section}
\numberwithin{definition}{section}
\numberwithin{theorem}{section}
\numberwithin{example}{section}
\numberwithin{lemma}{section}
\numberwithin{prop}{section}
\numberwithin{corollary}{section}
\numberwithin{problem}{section}
\numberwithin{remark}{section}
\usepackage{xpatch}

% \makeatletter
% \AtBeginDocument{\xpatchcmd{\@thm}{\thm@headpunct{.}}{\thm@headpunct{}}{}{}}
% \makeatother
\newcommand{\attvis}[2]{\definecolor{att}{rgb}{1, #2, #2} \colorbox{att}{#1}}


\setlist[enumerate,1]{label={(\roman*)}}
\DeclareUnicodeCharacter{2212}{-}
\DeclareRobustCommand{\mymarginpar}[1]{%
 \marginpar[\raggedleft#1]{\raggedright#1}}
\pagestyle{headings}
% \linespread{1.1}

\title{ Notes from Class \vspace{0.5pc} \\ \large{Probabilistic Techniques and Algorithms in Data Analysis}}
\author{Min Tang}
%\date{27. Oktober 2019}
\begin{document}
%\maketitle
%\tableofcontents 
\newpage
\setcounter{section}{1}
\setcounter{page}{2}
\section{Randomized Matrix Multiplication}
\mymarginpar{\tiny{1. Lecture\\ 18.10.2022}}
\paragraph{Motivations:}
\begin{itemize}
\item Basis transformation
\item Greedy algorithm (later)
\item Application to randomized SVD
\end{itemize}
Linear operations are the most basic function model $\rightarrow$ Problem of crucial importance. How do we multiply matrices $A\in \R^{m\times n}, B\in \R^{n\times p}$?
\begin{equation*}
\begin{pmatrix}
a_{11} & a_{12} \\
a_{21} & a_{22} \\
\end{pmatrix}
\begin{pmatrix}
b_{11} & b_{12} \\
b_{21} & b_{22} \\
\end{pmatrix} =
\begin{pmatrix}
a_{11} b_{11} + a_{12} b_{21} & ... \\
... & ... \\
\end{pmatrix}
\end{equation*}
%\vspace{-1pc}
\begin{algorithm}[H]
\For {$i=1$ to $m$}{
	\For {$j=1$ to $p$}{
		$(AB)_{ij}$ = 0 \\
		\For {$k=1$ to $n$} {
			$(AB)_{ij} = (AB)_{ij}+A_{ik}B_{kj}$
		}
	}
}
\caption{Naive matrix multiplication}
\end{algorithm}
$\leadsto$ Computational cost of $O(m\cdot n \cdot p).$ 
\begin{itemize}
\item \emph{Can we do that faster?} Yes, Strassen's algorithm (based on tensor representations) and follow-ups give exact product in fewer operations. \\
For $m=n=p$: best known order is $O(m^{2.37...})$ 
	\subitem $\rightarrow$ still large for $m$ large
	\subitem $\rightarrow$ does not compete with standard approach due to implementation aspects.
\item \emph{Can we make it still faster if we are OK with approximate solutions?}
	\subitem $\rightarrow$ approximate solution $\leadsto$ small \emph{error}  
	\subitem $\rightarrow$ randomized algorithm with small probability of failure. 
\item To do that, we need measure for size of error. $\rightarrow$ Norms on matrix space:
\begin{align*}
\|A\|_F = \sqrt{\operatorname{tr}A^*A} = \sqrt{\sum_{i,j} a_{ij}^2}\quad\quad \text{Frobenius norm}\\
\|A\| = \sup_{x:\|x\|=1} \|Ax\|_2 \quad\quad \text{Spectral norm}
\end{align*}
The Frobenius norm will be used as a measure for error.
\end{itemize}
\drawaline
\mymarginpar{\tiny{2. Lecture\\ 25.10.2022}}
\textbf{Observation}:
\begin{equation*}
AB=\sum_{i} \underbrace{A^{(i)}}_{i\text{-th column of }A} \cdot \underbrace{B_{(i)}}_{i\text{-th row of }B}
\end{equation*}
\textbf{Goal}: Approximate matrix product with fewer operations \\
\textbf{Idea}: Subsample and renormalize.
\begin{enumerate}
	\item \textbf{Subsampling}: sum only \textbf{random samples} of outer products, that is,
	$$S_1 = \sum_{t=1}^c A^{(i_t)}B_{(i_t)}$$
	where the $i_t$ are drawn independently at random according to some probability measure $\nu$ on $[n]:= \{1,...,n\}$. 
	\item \textbf{Renormalization:} Calculate 
	\begin{equation*}
	\begin{split}
		\E S_1 & = c \cdot  \E[A^{(i_1)} B_{(i_1)}]		\\
				& = c \cdot \sum_{j=1}^n \nu(j) A^{(j)} B_{(j)} 
	\end{split}
	\end{equation*}
		We would like this to be $AB$. So we renormalize and consider
	\begin{equation} \label{matrix-mult-random}
		\boxed{S= \sum_{t=1}^c \frac{1}{c\nu(i_t)}A^{(i_t)} B_{(i_t)}} \implies \E S= AB
	\end{equation}
\end{enumerate}
\begin{algorithm}[H]
\SetAlgoLined
\SetKwInOut{Input}{Input}\SetKwInOut{Output}{Output}
% initialization\;
 \Input{$A\in \R^{m \times n}, B\in \R^{n \times p}, c\in \N, p_1=\nu(1),...,p_n=\nu(n)$}
 \Output{Approximate product $P\in \R^{m \times p}$}
 \BlankLine
\For {$t=1$ to $c$}{
	Pick $i_t \in [n]$ i.i.d. according to $\nu$ \\
	Set $c^{(t)} = \frac{A^{(i_t)}}{\sqrt{c p_{i_t}}}$, $R_{(t)} = \frac{B_{(it)}}{\sqrt{cp_{i_t}}}$
}
\Return{$P = CR$ \small{(calculated using your favorite deterministic algorithm)}}
\label{algo2:random-matr-mult}
\caption{Randomized matrix multiplication}
\end{algorithm}
\begin{itemize}
	\item How should we choose the measure $\nu$?
		\subitem $\rightarrow$ Answer potentially depends on the error measure.
		\subitem  Here: difference in Frobenius norm $\|M\|_F^2 = \sum_{i,j}M_{ij}^2.$ 
	\item First step: Calculate the variance of the approximate products.
\end{itemize}

\begin{lemma}[Variance of the approximate products] \label{lemma:var-appro-prod}
\begin{mdframed}
Given matrices $A\in \R^{m \times n}$, $B\in \R^{n\times p}$, define $S$ as in \eqref{matrix-mult-random}. Then
\begin{equation}
\operatorname{Var} S_{ij} = \frac{1}{c} \sum_{k=1}^n \frac{A_{ik}^2 B_{kj}^2}{\nu(k)} - \frac{1}{c} (AB)_{ij}^2.
\end{equation}
\end{mdframed}
\begin{proof}
Fix $i,j$ and set, for $t\in [c]$,
\begin{equation*}
X_t := \left(
\frac{A^{(i_t)}B_{(i_t)}}{c\nu(i_t)}
\right)_{ij} = \frac{A_{i i_t}B_{i_t j}}{c\nu(i_t)}.
\end{equation*}
Thus
\begin{equation*}
\E X_t^2 = \sum_{k=1}^n \nu(k) \left(
\frac{A_{ik}B_{kj}}{c\nu(k)}
\right)^2 = \frac{1}{c^2}\sum_{k=1}^n \frac{1}{\nu(k)}A^2_{ik}B^2_{kj},
\end{equation*}
and similarly
\begin{equation*}
\E X_t = \sum_{k=1}^n \nu(k) \frac{A_{ik}B_{kj}}{c \nu(k)}= \frac{1}{c} (AB)_{ij}.
\end{equation*}
Note that this is consistent with $\E S= AB$. Moreover,
\begin{equation*}
\begin{split}
\E S_{ij}^2 = \E \left( \sum_{t=1}^c X_t \right)^2 & \overset{\text{indep.}}{=} \sum_{t=1}^c \E X_t^2 + \sum_{t_1,t_2, t_1 \neq t_2}^c \E X_{t_1} \E X_{t_2} \\
& = \frac{1}{c}\sum_{k=1}^n \frac{1}{\nu(k)} A_{ik}^2 B_{kj}^2 + \underbrace{c(c-1)}_{\text{summands}} \left(\frac{1}{c}(AB)_{ij}\right)^2 \\
& =  \frac{1}{c}\sum_{k=1}^n \frac{1}{\nu(k)} A_{ik}^2 B_{kj}^2 + \frac{c-1}{c} (AB)_{ij}^2.
\end{split}
\end{equation*}
Consequently,
\begin{equation*}
\operatorname{Var}S_{ij} = \E S_{ij}^2 - \underbrace{(\E S_{ij})^2}_{(AB)_{ij}^2} =  \frac{1}{c}\sum_{k=1}^n \frac{1}{\nu(k)} A_{ik}^2 B_{kj}^2 - \frac{1}{c} (AB)_{ij}^2,
\end{equation*}
which proves the lemma.
\end{proof}
\end{lemma}
\begin{lemma}[Expected error of approximate products] \label{lemma-expected-error-approx-prod}
\begin{mdframed}
Given matrices $A\in\R^{m \times n}$, $B \in \R^{n\times p}$, define the approximate product $S$ as in \eqref{matrix-mult-random}, then
\begin{equation} \label{eq:expected-error}
{
\E[\|AB-S\|_F^2]= \sum_{k=1}^n \frac{\|A^{(k)}\|_2^2 \|B_{(k)}\|_2^2}{c\nu(k)} - \frac{1}{c} \|AB\|_F^2}.
\end{equation}
\end{mdframed}
\begin{proof}
Note that
\begin{equation*}
\begin{split}
\E[\|AB-S\|_F^2] & = \sum_{i=1}^m \sum_{j=1}^p \E [(AB-S)_{ij}^2] \\
& = \sum_{i=1}^m \sum_{j=1}^p \underbrace{\E [(S-\E S)^2_{ij}}_{\operatorname{Var}S_{ij}}].
\end{split}
\end{equation*}
Thus from Lemma \ref{lemma:var-appro-prod}, it follows that
\begin{equation*}
\begin{split}
\E[\|AB-S\|_F^2] & =  \frac{1}{c}\sum_{k=1}^n \frac{1}{\nu(k)} \left( \sum_{i=1}^m A_{ik}^2 \right)
\left( \sum_{j=1}^p B_{kj}^2  \right)- \frac{1}{c} \sum_{i,j}(AB)_{ij}^2 \\
&= \sum_{k=1}^n \frac{\|A^{(k)}\|_2^2 \|B_{(k)}\|_2^2}{c\nu(k)} - \frac{1}{c} \|AB\|_F^2
\end{split}
\end{equation*}
which completes the proof.
\end{proof}
\end{lemma}
Best we can hope for: the error $\|AB-S\|_F^2$ is consistently close to expectation (``concentration phenomenon''). Thus we should choose the measure $\nu$ such that 
\begin{equation*}
 \sum_{k=1}^n \frac{\|A^{(k)}\|_2^2 \|B_{(k)}\|_2^2}{c\nu(k)}
\end{equation*}
is small. (Note that the second summand is independent of measure $v$.) \\
Let $p$ denote the vector of mass distribution of $\nu$, i.e., $p_k=\nu(k)$. Then we are looking for the minimizer of the smooth function
\begin{equation*}
f(p) = \sum_{k=1}^n \frac{\|A^{(k)}\|_2^2 \|B_{(k)}\|_2^2}{p_k}
\end{equation*}
over
$$
P:=\bigg\{p \in [0,1]: \underbrace{\sum_{k=1}^n p_k}_{=: g(p)} =1\bigg\}.
$$
Lagrange multiplier formulation: Extremum must satisfy, for some $\lambda \in \R$
\begin{equation*}
\begin{split}
0 & = \nabla f(p) + \lambda \nabla g(p) \\
& = \left(
- \frac{\|A^{(k)}\|_2^2 \|B_{(k)}\|_2^2}{p_k^2} 
\right)_{k=1}^n + \lambda (\underbrace{1,...,1}_{n \text{ times}})
\end{split}
\end{equation*}
This implies
\begin{equation*}
\lambda p_k^2 = \|A^{(k)}\|_2^2 \|B_{(k)}\|_2^2.
\end{equation*}
Thus,
$$
\sqrt{\lambda} \overset{\sum_{k}p_k=1}{=} \sum_{k=1}^n \sqrt{\lambda}p_k = \sum_{k=1}^n  \|A^{(k)}\|_2 \|B_{(k)}\|_2.
$$
If that sum vanishes, the product $AB$ vanishes. If not, we obtain the unique critical point
\begin{equation} \label{eq:opt-prob}
\boxed{
p = \left( 
\frac{\|A^{(j)}\|_2 \|B_{(j)}\|_2}{\sum_{k=1}^n \|A^{(k)}\|_2 \|B_{(k)}\|_2}
\right)_{j=1}^n
}.
\end{equation}
If $p_k \to 0$ for some $k$ with $\|A^{(k)}\|\neq 0 \neq \|B_{(k)}\|_2$, then $f(p) \to \infty$, so the critical point must be the minimizer. \\
Choosing $\nu(j) = p_j$, we obtain
\begin{equation} \label{eq:optimal-prob-error-est}
\boxed{\E [\|AB-S\|_F^2] = \frac{1}{c} \left( \sum_{k=1}^n \|A^{(k)}\|_2 \|B_{(k)}\|_2\right)^2- \frac{1}{c} \|AB\|_F^2}.
\end{equation}
\begin{remark*}
For rank 1 matrices, it holds that
\begin{equation}\label{eq:ident-rank-1-matr}
\boxed{\|A^{(k)}\|_2 \|B_{(k)}\|_2 = \|A^{(k)}B_{(k)}\| = \|A^{(k)}B_{(k)}\|_F}.
\end{equation}
The interpretation of optimal sampling strategy: we bias the random samples towards rank one components which are larger in norm.%\\
%\drawaline 
\end{remark*}
In our \textbf{implementation}, we have two different objectives: \vspace{-0.5pc}
\begin{itemize}[itemsep=0pt]
	\item minimize number of mathematical operations
	\item minimize storage space requirements
\end{itemize}
In addition to carrying out the reduced size multiplication [$O(m\cdot c \cdot p)$ operations], we need to compute \eqref{eq:opt-prob} (i.e., the probability measure $\nu$). We need
\begin{itemize}
	\item $O(m)$ operations to compute each $\|A^{(n)}\|_2$,
	\item $O(p)$ operations to compute each $\|B_{(n)}\|_2$,
\end{itemize}
and in total $O(mn+np)$. \\\\
\emph{Space requirement:} $O(n)$ to store all of them.\\\\
How do we then \emph{sample}? We need a model for accessing data. \emph{Model}: Data streaming, i.e., data ``stream by'', one computes on the stream without storing. $\to$ ``pass efficiency''.\\\drawaline
\mymarginpar{\tiny{3. Lecture\\ 08.11.2022}}
\textbf{Goal:} Approximate $AB$ by
\begin{equation*}
\sum_{k=1}^c \frac{1}{c\nu(i_k)} A^{(i_k)} B_{(i_k)} =: S,
\end{equation*}
where $i_k$ are i.i.d. drawn from probability measure $\mu$.\vspace{0.5pc} \\
\textbf{Optimal choice:}
\begin{equation*}
\nu(k) = \frac{\| A^{(k)}\|_2 \cdot \|B_{(k)}\|_2}{\sum_{j=1}^n \|A^{(j)}\|_2 \|B_{(j)}\|_2}.\vspace{0.5pc}
\end{equation*}
\textbf{Next:} Model for accessing data.
\begin{definition}[Pass efficient model]
\begin{mdframed}
In the pass efficient model, the only access an algorithm has to the data is via a pass, i.e., a sequential read of the entire data set. An algorithm is \emph{pass-efficient}, if it requires a small constant number of passes and additional space and time, which are \emph{sub-linear} in the length of the data stream.
\end{mdframed} \vspace{-1pc}
\begin{itemize}[-,itemsep=0pt]
\item idealized model
\item \emph{Motivation:} In many applications, one has the ability to store or generate larger amounts of data, but has random access to only linked amounts. 
\end{itemize}
$g$ is sublinear in $N \iff g(N) = o(N) \iff \forall C >0 \, \exists N_0 \in \N \, \forall N \geq N_0: g(N) \leq CN$. 
\end{definition}
\begin{remark*}
As we ``do something'' whenever we read an entry, the number of operations is in fact linear, main requirement: sublinear access queries.
\end{remark*}
\begin{algorithm}[H]
\SetAlgoLined
\SetKwInOut{Input}{Input}\SetKwInOut{Output}{Output}
% initialization\;
 \Input{$\{a_1,...,a_n\}, a_i \geq 0, \sum_i a_i > 0$}
 \Output{$i^*,a_{i^*}$}
 \BlankLine
$D= 0 \quad \leftarrow$ normalization factor \\
\For {$i=1$ to $n$}{
	$D = D+a_i$ \\
	with probability $\frac{a_i}{D}$ let $i^*=i, a_{i^*} = a_i$ (or w.p. 1 if $a_i=D=0$)
}
\Return{$i^*, a_{i^*}$}
\caption{Select algorithm}
\end{algorithm}
\begin{remark*}
Indeed, this algorithm returns each $i$ with probability $\frac{a_i}{\sum_j a_j}$.
\end{remark*}
\begin{example*}
Let $a_1=a_2=a_3 =5$. Then,
\begin{itemize}[itemsep=0pt]
	\item $i=1$: $D=5$ with $P=\frac{5}{5}=1$. Choose $i^* =1$.
	\item $i=2$: $D=10$ with $P=\frac{5}{10}=\frac{1}{2}$. Then $P(i^*=1)=P(i^*=2)=\frac{1}{2}$. Choose $i^* = 2$.
	\item $i=3$: $D=15$ with $P=\frac{5}{10}=\frac{1}{3}$. Then by independence, $P(i^*=1)=P(i^*=2)==P(i^*=3)=\frac{1}{3}$. Choose $i^* = 3$.
\end{itemize} 
\end{example*}
\begin{lemma}
\begin{mdframed}
Suppose that the selection algorithm is applied with inputs $\{a_1,...,a_n\}$, $a_i \geq 0$. Then the additional storage space required is $O(1)$ and the output $i^*$ satisfies $P(i^*=i) = \frac{a_i}{\sum_{j=1}^n a_j}$.
\end{mdframed}
\begin{proof}
Note that only the current values of $i^*, a_{i^*},$ and $D$ must be retained, which corresponds to $O(1)$ space.\\
The reminder of the proof is by induction. After reading $a_1$, one has $i^*=1$ w.p. $\frac{a_1}{a_1}=1$ (provided $a_1=0$, otherwise 1 by definition). As the induction hypothesis, assume that after reading $a_1,...,a_l$, the variable $i^*$ satisfies $$P(i^*=i) = \frac{a_i}{D_l}\quad \text{for } i \in [l],$$
where $D_l = \sum_{i=1}^l a_i$. Upon reading $a_{l+1}$, the algorithm sets $i^*=l+1$ w.p. $\frac{a_{l+1}}{D_{l+1}}$ and retains its previous value w.p. $1-\frac{a_{l+1}}{D_{l+1}}  = \frac{D_l}{D_{l+1}}$. Thus by independence, after reading $a_1,...,a_{l+1}$, one has for $i< l+1$: $$P(i^*=i)=\frac{a_i}{D_l}\cdot \frac{D_l}{D_{l+1}}=\frac{a_i}{D_{l+1}}.$$ and for $i=l+1$, the same holds by construction. This completes the induction step and the results follows as $\frac{a_i}{D_n} = \frac{a_i}{\sum_{j=1}^n a_j}$.
\end{proof}
\end{lemma}
To draw $c$ independent samples, the select algorithm is repeated $c$ times. Thus the total number of operations required to select $c$ indices independently at random is $O(c \cdot n)$, where $c$ is the number of passes and $n$ is the number of elements/operations per pass. $\leadsto$ Total number of operations $O(mn + np + cn +mcp)$.
\paragraph{Observation:} The only cubic term in the above expression is $m c  p$. Thus, $c$ determines the number of operations. To determine the size of $c$, we, however, face the following conflicting goals: \vspace{-0.5pc}
\begin{itemize}[-,itemsep=0pt]
	\item Number of summands $c$ should be as small as possible.
	\item Error ``in most cases'' should be as small as possible.
	\item Probability of exceptional cases (``failure'') should be as small as possible.
\end{itemize}
Those goals can be reformulated as archiving 
\begin{equation}
\|AB-S\|_F \leq \varepsilon \|A\|_F \|B\|_F \quad \text{w.p. } \geq 1-\delta
\end{equation}
for $\varepsilon$, $\delta$ and $c$ as small as possible.
\begin{remark*}
Assuming that $S=CR \in \R^{m\times p}$ is the output of Algorithm \ref{algo2:random-matr-mult}\ for the parameter $c$ with $$c\geq \frac{1}{\delta^2\varepsilon^2}$$ and optimal probability defined in \eqref{eq:opt-prob}, then it holds 
\begin{equation} \label{eq:error-bound-epsilon-f}
\|AB-S\|_F \leq \varepsilon \|A\|_F \|B\|_F\quad \text{w.p. } \geq 1 - \delta
\end{equation}
for arbitrary $\varepsilon,\delta>0$.
\begin{small} 
\begin{proof}(Exercise 3.2) First, Markov inequality implies
$$
P(\|AB-S\|_F \geq \varepsilon \|A\|_F \|B\|_F) \leq \frac{\E [\|AB-S\|_F]}{\varepsilon \|A\|_F\|B\|_F}.
$$
We now estimate the numerator $\E [\|AB-S\|_F]$
\begin{equation*}
\begin{split}
\E[\|AB-S\|_F^2] & \underset{\text{opt. prob.}}{\overset{\eqref{eq:optimal-prob-error-est}}{=}} \frac{1}{c} \underbrace{\left( \sum_{k=1}^n \|A^{(k)}\|_2 \|B_{(k)}\|_2\right)^2}_{\overset{\text{C.S.}}{\leq}
(\sum_{k=1}^n \|A^{(k)}\|_2^2)(\sum_{k=1}^n \|B_{(k)}\|_2^2)
}- \frac{1}{c} \|AB\|_F^2 \\
& \leq \frac{1}{c} \left(\sum_{k=1}^n \|A^{(k)}\|_2^2\right)\left(\sum_{k=1}^n \|B_{(k)}\|_2^2\right)  \\
& = \frac{1}{c}\|A\|_F^2 \|B\|_F^2.
\end{split}
\end{equation*}
Therefore,
\begin{equation*}
\begin{split}
P(\|AB-S\|_F \geq \varepsilon \|A\|_F \|B\|_F) &\leq \frac{\E [\|AB-S\|_F]}{\varepsilon \|A\|_F\|B\|_F} \\
&\leq \frac{1}{\sqrt{c}\varepsilon} \overset{c \geq \frac{1}{\delta^2\varepsilon^2}}{\leq} \delta,
\end{split}
\end{equation*}
and  thus $P(\|AB-S\|_F \leq \varepsilon \|A\|_F \|B\|_F) \geq 1-\delta$.
\end{proof}
\end{small}
\end{remark*}
This result may suffice if 10\% failures are fine. But for very small probabilities of failure, we need very large number of measurements [see Exercise 3.2 (c)].
But why can we hope for something better?%\vspace{-0.5pc}
\begin{itemize}[itemsep=0pt]
	\item $S$ is average of independent random matrices.
	\item Law of large numbers suggests convergence to the mean, which is $AB$. 
\end{itemize} \newpage
However, we still face two issues:%\vspace{-0.5pc}
\begin{enumerate}[itemsep=0pt]
	\item We need non-asymptotic behavior rather than limit as $n\to\infty$ (as in LLN).
	\item We need matrix version in Frobenius norm.
\end{enumerate}
\textbf{Idea:} Do not look at this as a matrix problem, rather consider the real-valued function
\begin{equation} \label{eq2.9:functionf}
\boxed{F:(i_1,...,i_c) \mapsto \underbrace{\left\|
\sum_{t=1}^c \frac{1}{c\nu(i_t)}A^{(i_t)} B_{(i_t)} - \E \left[\sum_{t=1}^c \frac{1}{c\nu(i_t)}A^{(i_t)} B_{(i_t)}
\right] \right\|_F}_{\|S-AB\|_F}}.
\end{equation}

\textbf{Observations:}%\vspace{-0.5pc}
\begin{itemize}[-,itemsep=0pt]
\item Inputs are i.i.d. random variables with values in $[n]$.
\item $\E[F]$ is under control (via Lemma \ref{lemma-expected-error-approx-prod}).
\end{itemize}
\textbf{Tool:} Version of Mcdiarmid's inequality. 
\mymarginpar{\tiny{4. Lecture \\ 22.11.2022}} %\vspace{-1pc}
\begin{theorem}[Mcdiarmid's inequality] \label{theorem:mcdiarmid}
\begin{mdframed}
Let $S\subset \N$ and let $X_1,...,X_n$ be independent random variables with values in $S$. Let $F:S^n \to \R$ be a map fulfilling the following \emph{bounded differences property}: There exists $\Delta >0$ such that for all $x_1,...,x_n,x_1',...,x_n'$, it holds that \vspace{-0.5pc}
\begin{equation} \label{eq2.10:bounded-diff-prop}
|F(x_1,...,x_n) - F (x_1',...,x_n')| \leq \Delta \sum_{i=1}^n \mathbbm{1}_{\{x_i \neq x_i'\}}.\vspace{-0.5pc}
\end{equation}
Then for any $t>0$, \vspace{-0.5pc}
\begin{equation}
P(F(X_1,...,X_n) - \E [F(X_1,...,X_n] \geq t) \leq \exp \left(
\frac{-2t^2}{n\Delta^2}.
\right)
\end{equation}
\end{mdframed}
\begin{proof}
See Exercise 3.3.
\end{proof}
\end{theorem}
With this tool, we can show the following bound with a much better scaling in $\delta$.
\begin{theorem}
\begin{mdframed}
Suppose $A\in \R^{m\times n}, B\in \R^{n\times p}, c\in [n]$, and $\nu$ is a probability measure on $[n]$ such that, for some constant $0<\beta\leq 1$
\begin{equation}\label{eq:bound-for-measure-nu}
\nu(k) \geq \frac{\beta \|A^{(k)}\|_2\|B_{(k)}\|_2}{\sum_{k'=1}^n \|A^{(k')}\|_2\|B_{(k')}\|_2 }.
\end{equation}
We construct $S$ using Algorithm \ref{algo2:random-matr-mult} above. Then for all $\delta \in (0,1)$ and the associated $\eta = 1+ \sqrt{\frac{2}{\beta} \log(\frac{1}{\delta})}$ it holds that 
\begin{equation}
\|AB-S\|_F \leq \frac{\eta}{\sqrt{\beta c}} \|A\|_F \|B\|_F \quad	\text{w.p. } \geq 1-\delta.
\end{equation}
The role of $\beta$ is to measure the amount of deviation from the optimal sampling density.
\end{mdframed}
\begin{proof}
First note that 
\begin{equation}
\begin{split}
\E \|AB - S\|_F & \overset{\text{Jensen}}{\leq} \sqrt{\E \|AB-S\|_F^2}  \\
& \overset{\eqref{eq:expected-error}}{=} 
\sqrt{\sum_{k=1}^n \frac{\|A^{(k)}\|_2^2 \|B_{(k)}\|_2^2}{c\nu(k)} - \frac{1}{c} \underbrace{\|AB\|_F^2}_{\geq 0}} \\
& \overset{\eqref{eq:bound-for-measure-nu}}{\leq} \frac{1}{\sqrt{\beta c}} \underbrace{\sum_{k=1}^n \|A^{(k)}\|_2 \|B_{(k)}\|_2}_{= \langle (\|A^{(k)}\|_2)_{k=1}^n, (\|B_{(k)}\|_2)_{k=1}^n \rangle} \\
& \overset{\text{C.S.}}{\leq} \frac{1}{\sqrt{\beta c}} 
\underbrace{\left(
 \sum_{k=1}^n \|A^{(k)}\|_2^2
\right)^\frac{1}{2}}_{\|A\|_F}
\underbrace{\left(
 \sum_{k=1}^n \|B_{(k)}\|_2^2
\right)^\frac{1}{2}}_{\|B\|_F} = \frac{1}{\sqrt{\beta c}} \|A\|_F\|B\|_F.
\end{split}
\end{equation}
Thus it remains to show that
\begin{equation*}
\|AB-S\|_F - \E[\|AB-S\|_F] \leq \frac{\eta -1}{\sqrt{\beta c}} \|A\|_F \|B\|_F\quad \text{w.p. } \geq 1-\delta.
\end{equation*}
To prove this, we apply Theorem \ref{theorem:mcdiarmid} with \eqref{eq2.9:functionf}:
$$F:(i_1,...,i_c) \mapsto \left\|
\sum_{t=1}^c \frac{1}{c\nu(i_t)}A^{(i_t)} B_{(i_t)} - AB \right\|_F,$$
and $\hat{n}=c$,\footnote{Here $\hat{n}$ refers to the $n$ from Theorem \ref{theorem:mcdiarmid}.} $S=[n].$ We first need to check the bounded difference property \eqref{eq2.10:bounded-diff-prop}. We fix $r \in [c]$. Then
\begin{equation*}
\begin{split}
& |F(i_1,...,i_r,i_{r+1},...,i_c)-F(i_1,...,i_r',i_{r+1},...,i_c)|\\
= &\left|
\left\|
\frac{1}{c}
\sum_{t=1}^c \frac{1}{\nu(i_t)} A^{(i_t)} B_{(i_t)} - AB
\right\|_F - 
\left\|
\frac{1}{c}
\sum_{t=1, t\neq r}^c \frac{1}{\nu(i_t)} A^{(i_t)} B_{(i_t)} - AB +
\frac{1}{c} \frac{1}{\nu(i'_r)}A^{(i_r')}B_{(i'_r)}
\right\|_F
\right| \\
\overset{\text{rev. }\Delta\text{-ineq}}{\leq}
&
\left\|
\frac{1}{c\nu(i_r)} A^{(i_r)} B_{(i_r)} - \frac{1}{c\nu(i'_r)} A^{(i'_r)} B_{(i'_r)}
\right\|_F \\
\underset{\eqref{eq:ident-rank-1-matr}}{\overset{\Delta\text{-ineq}}{\leq}} &
\frac{1}{c\nu(i_r)} \|A^{(i_r)}\|_2 \|B_{(i_r)}\|_2 +  \frac{1}{c\nu(i'_r)} \|A^{(i'_r)}\|_2 \|B_{(i'_r)}\|_2 \\
\leq \ \ & \frac{2}{c} \max_{\alpha \in [n]}\frac{\|A^{(\alpha)}\|_2\|B_{(\alpha)}\|_2}{\nu(\alpha)} \\
\overset{\eqref{eq:bound-for-measure-nu}}{\leq} \, &
\frac{2}{c\beta} \sum_{k'=1}^n \|A^{(k')}\|_2 \|B_{(k')}\|_2 \overset{\text{C.S.}}\leq \underbrace{\frac{2}{\beta c}\|A\|_F \|B\|_F}_{=: \Delta}.
\end{split}
\end{equation*}
By the triangle inequality,this implies the bounded difference inequalities for arguments that differ in more than one entry and the same $\Delta$. The result follows from Theorem \ref{theorem:mcdiarmid} noting that $\exp(-\frac{\beta}{2}(\eta-1)^2)=\delta$.
\end{proof}
\end{theorem}
\section{Randomized Principal Component Analysis}
\textbf{Motivation.} Many high-dimensional data sets are intrinsically low-dimensional, i.e., they can be well approximated by a lower dimensional subspace. For a given matrix $A\in\R^{m \times n}$, we want to find a lower dimensional embedding $\tilde{A}_k$ of rank $k$ close to $A$ (cf. principal component analysis). 

\begin{definition}[Best $k$-rank approximation]
\begin{mdframed}
Given $A\in \R^{m \times n}, A_k \in \R^{m\times n}$ is a \emph{best rank-$k$ approximation to $A$} with respect to a norm $\vvvert \cdot \vvvert$ if $A_k\in \underset{B\in \R^{m \times n}, \operatorname{rk} B\leq k}{\operatorname{argmin}}\vvvert A-B\vvvert$.
\end{mdframed}
\end{definition} \vspace{-1pc}
\begin{remark*} With respect to the spectral norm and the Frobenius norm, $A_k$ can be computed via the singular value decomposition. If the SVD of $A$ is 
$$
U\begin{pmatrix}
\sigma_1 &    			&    & 0 \\
	           &\sigma_2	&	  &	\\
	           &				&	\ddots & \\
	0           &				&	 & \sigma_r
\end{pmatrix}V^*,
$$
then the matrix
$$
A_k = U\begin{pmatrix}
\sigma_1 &    			&    & 0 \\
	           &\ddots	&	  &	\\
	           &				&	\sigma_k & \\
	0           &				&	 & 0
\end{pmatrix}V^*
$$
is the best rank-$k$ approximation to $A$.
\end{remark*}
However, to compute the full SVD of $A$, we have a complexity of $O(\min(m^2n,mn^2))$, i.e. is cubic. Note that to archive our goal (finding $\tilde{A}_k$ of rank $k$ such that $A-\tilde{A}_k$ and $A-A_k$ are comparable in size (up to constant)), a \emph{crucial step} is to find left singular vectors $u_1,...,u_k$, i.e. matrix $U_k$, corresponding to $\sigma_1,...,\sigma_k$. To reduce complexity, we could first \emph{project} $A$, then compute SVD of the smaller matrix, and infer approximation of $U_k$. For \emph{projection}, we will use \emph{randomized Hadamard transform}.

\begin{definition}[Hadamard transform]\label{def:hadamard}
\begin{mdframed}
Let $n\in\N$ be a power of 2. Then the $n\times n$ \textbf{Hadamard matrix} $\widetilde{H}_n$ is defined recursively as follows 
\begin{equation}
\widetilde{H}_n = 
\begin{pmatrix}
\widetilde{H}_{\frac{n}{2}} & \widetilde{H}_{\frac{n}{2}} \\
\widetilde{H}_{\frac{n}{2}} & - \widetilde{H}_{\frac{n}{2}}
\end{pmatrix}
\quad \text{with}\quad
\widetilde{H}_2 = 
\begin{pmatrix}
+1 & +1 \\
+1 & -1
\end{pmatrix}.
\end{equation}
The \textbf{normalized Hadamard transform} is \vspace{-0.6pc} $$H=H_n =\frac{1}{\sqrt{n}} \widetilde{H}_n.\vspace{-0.6pc} $$
Furthermore, for a random diagonal matrix $D\in \R^{n\times n}$ with independent entries $D_{ii}$ with $P(D_{ii}=1)=P(D_{ii}=-1)=\frac{1}{2}$, $HD$ is called the \textbf{randomized Hadamard transform}.
\end{mdframed}
\end{definition}

\begin{remark*}
The Hadamard transform has following properties:\vspace{-0.5pc}
\begin{enumerate}[itemsep=0pt]
\item The randomized Hadamard transform $HD$ is orthogonal, because the normalized Hadamard transform $H$ and the Rademacher matrix $D$ are orthogonal [Ex. 5.1].
\item Computing $Hx$ (and hence $HDx$ for $x$ fixed needs $O(n\log n)$ operations (similar to fast Fourier transform, Hadamard $\hat{=}$ Fourier transform in $\mathbb{Z}_2$).
\end{enumerate}
\end{remark*}
Intuition for Hadamard transform: $HD$ ``spreads out'' mass/energy. We hope that after applying $HD$, it suffices to subsample according to uniform distribution (hence independent of $A$). This is made precise by the following lemma:
%\drawaline
\mymarginpar{\tiny{5. Lecture \\ 29.11.2022}} %\vspace{-1pc}
\begin{lemma}
\begin{mdframed}
Let $U\in \R^{n \times d}$ be a matrix with orthogonal columns and let the product $HD$ be the $n\times n$ randomized Hadamard transform as introduced in Definition \ref{def:hadamard}. Then, it holds that
\begin{equation*}
\|(HDU)_{(i)}\|_2^2 \leq \frac{2d\log(40nd)}{n}\ \ \forall i=1,...,n		\quad \text{w.p. } \geq 1-\frac{1}{20}.
\end{equation*}
\end{mdframed}
\begin{proof}
Consider $(HDU)_{ij}$ for some $(i,j)\in [n]\times [d]$. As $D$ is diagonal, one has
\begin{equation*}
(HDU)_{ij} = \sum_{l=1}^n H_{il}D_{ll}U_{lj} = \sum_{l=1}^n D_{ll}(H_{il}U_{lj}).
\end{equation*}
As $D_{ll}$ are i.i.d. random signs, $|D_{ll}(H_{il}U_{lj})|\leq |(H_{il}U_{lj})|$ a.s., and $\E[D_{ll}(H_{il}U_{lj})] = 0$, Hoeffding's inequality implies for all $t>0$
\begin{equation*}
\begin{split}
P\left( \left|
\sum_{l=1}^n D_{ll}(H_{il}U_{lj})
\right| \geq t \right)  
\leq & \ 2 \exp \left( -
\frac{t^2}{2\sum_{l=1}^n(H_{il}U_{lj})^2}
\right)\\
\overset{(*)}{=} & \  2 \exp \left( 
- \frac{nt^2}{2}
\right),
\end{split}
\end{equation*}
where $(*)$ comes from the fact
\begin{equation*}
\sum_{l=1}^n (H_{il}U_{lj})^2 \overset{|H_{il}|=\frac{1}{\sqrt{n}}}{=} \frac{1}{n}\sum_{l=1}^n U_{lj}^2 = \frac{1}{n}\|U^{(j)}\|_2^2 \overset{U \text{ orth.}}{=} \frac{1}{n}.
\end{equation*}
Consider 
\begin{equation*}
\delta = 2 \exp \left( 
- \frac{nt^2}{2}
\right)
\end{equation*}
and choose $t:= \sqrt{\frac{2\log\left(\frac{2}{\delta}\right)}{n}}$,
we get
\begin{equation}\label{eq:help}
P\left( \left|
(HDU)_{ij}
\right| \geq \sqrt{\frac{2\log\left(\frac{2}{\delta}\right)}{n}} \right)  =
P\left( \left|
\sum_{l=1}^n D_{ll}(H_{il}U_{lj})
\right| \geq \sqrt{\frac{2\log\left(\frac{2}{\delta}\right)}{n}} \right)  
\leq \delta.
\end{equation}
Choosing $\delta = \frac{1}{20nd}$, we obtain
\begin{equation*}
\begin{split}
& P\left(
\exists i \in [n]: \|(HDU)_{(i)}\|_2^2 \geq \frac{2d\log(40nd)}{n}
\right) \\
\overset{(*)}{\leq} \ & P \left(
\exists (i,j): \left|
(HDU)_{ij}%\sum_{l=1}^n D_{ll}(H_{il}U_{lj})
\right| \geq \sqrt{\frac{2\log(40nd)}{n}}
\right) \\
\overset{\text{Subadd.}}{\leq} \ &
\sum_{i=1}^n \sum_{j=1}^d P\left( |(HDU)_{(ij)}| \geq \sqrt{\frac{2\log(40nd)}{n}}
\right) \\
\underset{\frac{2}{\delta}=40nd}{\overset{\eqref{eq:help}}{\leq}} \ &\frac{1}{20nd} \leq \frac{1}{20},
\end{split}
\end{equation*}
where $(*)$ comes from the fact 
\begin{equation*}
\left(
\forall (i,j): \left|
(HDU)_{ij}%\sum_{l=1}^n D_{ll}(H_{il}U_{lj})
\right| \leq \sqrt{\frac{2\log(40nd)}{n}}
\right)  \implies \left(
\forall i \in [n]: \|(HDU)_{(i)}\|_2^2 \leq \frac{2d\log(40nd)}{n}
\right).
\end{equation*}
Considering the complement yields claim.
\end{proof}
\begin{remark*}
The result holds even for arbitrary $U\in \R^{n\times D}$ (not necessarily orthogonal), see Exercise 5.2.
\end{remark*}
\end{lemma}
\underline{Idea:} Apply randomized Hadamard transformation from one side, e.g., postmultiplying $A\in\R^{m\times n}$ by $(HD)^T$, forming $ADH\in \R^{m \times n}$. Then sample (uniformly at random) $c$ columns from $ADH$, thus forming a smaller matrix $C\in\R^{m\times c}$. From $C$, construct approximations of the top $k$ singular values of $A$.\\

\begin{algorithm}
\SetAlgoLined
\SetKwInOut{Input}{Input}\SetKwInOut{Output}{Output}
% initialization\;
 \Input{$A\in\R^{m\times n}$, rank parameter $k \ll \min\{m,n\}$, error parameter $\varepsilon \in \left( 0, \frac{1}{2}\right)$, name of samples $c\in \N$.}
 \Output{$\widetilde{U}_k \in \R^{m\times k}$}
 \BlankLine
 set $S = 0^{n \times c}$ (all entries are $0$) \textit{//sample matrix}\\
 \For {$t=1$ to $c$}{ 
	\emph{// i.i.d. trials with replacement} \\
	select uniformly at random $i_t \in [n]$ and set $S^{(t)} = \sqrt{\frac{n}{c}} e_{i_t}$
}
compute $C=ADHS$ where $HD$ is the randomized Hadamard transform\\
compute $U_C$, an ONB for the column space of $C$ (e.g., via SVD)\\
compute $W=U_C^TA$ and compute its top $k$ singular vectors $U_{W,k}$ (or less if $\operatorname{rank} W <k$) \\
\Return{$\widetilde{U}_k = U_CU_{W,k} \in \R^{m \times k}$}
\caption{Randomized PCA} \label{algo:random-pca}
\end{algorithm}
\newpage
\begin{theorem}
\begin{mdframed}
Let $A\in\R^{m\times n}$, let $k$ be a rank parameter, and let $\varepsilon\in (0,\frac{1}{2})$. If we set $$ c \geq c_0 \frac{k\log n}{\varepsilon^2} \left(
\log\frac{k}{\varepsilon^2} + \log \log n
\right)$$
for a fixed constant $c_0$, then with probability $\geq 0.85$, Algorithm \ref{algo:random-pca} returns a matrix $\widetilde{U}_k \in \R^{m\times k}$ such that
\begin{equation*}
\|A-\widetilde{U}_k\widetilde{U}_k^TA\|_F \leq (1+\varepsilon) \|A-A_k\|_F.
\end{equation*}
The running time of the algorithm is $O(mnc)$. (Recall: $A_k$ denotes the best $k$-rank approximation of $A$.)
\end{mdframed}
\begin{remark}
Repeating Algorithm \ref{algo:random-pca} $\lceil \log \frac{1}{\delta}/ \log C(\delta) \rceil$-times and keeping the matrix $\widetilde{U}_k$ that minimizes $\|A-\widetilde{U}_k\widetilde{U}_k^TA\|_F$ reduces the failure probability to at most $\delta$ for $\delta \in (0,1)$.
\end{remark}
\textbf{Proof strategy of Theorem 3.4.} We split $A = A_k + A_{k,\perp}$ for $A_k$ the best rank-$k$ approximation. We first show 
\begin{equation}
\|A-\widetilde{U}_k \widetilde{U}_k^T A \|_F^2 \leq \|A_k - U_CU_C^TA_k\|_F^2 + \|A_{k,\perp}\|_F^2
\end{equation}
So in the reminder, we ``only'' need to control the first term on the right hand side. (The second yields $\|A-A_k\|_F$ on the right hand side of the statement we aim to prove. In order to show this inequality, we need the following lemma.
\begin{lemma}
Let $U_C$ and $\widetilde{U}_k$ be as in Algorithm \ref{algo:random-pca}, i.e. the columns of $U_C$ are an ONB of $\operatorname{range}(C)$.  Then
\begin{equation}
A-\widetilde{U}_k\widetilde{U}_k^T A = A-U_C(U_C^TA)_k
\end{equation}
In addition, $U_C(U_C^TA)_k$ is the best rank-$k$ approximation of $A$ with respect to $\|\cdot\|_F$ that lies within $\operatorname{range}(C)$, i.e.,
\begin{equation} \label{lem:3.6-2}
\|A-U_C(U_C^TA)_k\|_F^2 = \min_{\operatorname{rank}(Y)\leq k} \|A-U_CY\|_F^2
\end{equation}
\begin{proof}
Recall that $\widetilde{U}_k = U_C U_{W,k}$, where $U_{W,k}$ is the matrix of the top $k$ left singular vectors of $W = U_C^TA$. In other words, if $W=U_W\Sigma_WV_W^T$ denotes the SVD of $W$, the best $k$-term approximation $W_k$ of $W$ with respect to $\|\cdot \|_F$ has a singular value decomposition
\begin{equation}
W_k = U_{W,k} \Sigma_{W,k} V_{W},
\end{equation}
where $\Sigma_{W,k}$ consists of the first $k$ rows of $\Sigma_W$. Consequently,
\begin{equation}
\begin{split}
A-\widetilde{U}_k\widetilde{U}_k^TA = & A-U_CU_{W,k}U_{W,k}^TU_C^TA \\
= & A- U_C U_{W,k} \underbrace{\underbrace{U_{W,k}^T U_W}_{(\id_k \mathbf{0})}\Sigma_W}_{
\Sigma_{W,k}
}V_W \\
= & A-U_CW_k \\
= & A-U_C (U_C^TA)_k.
\end{split}
\end{equation}
To prove \eqref{lem:3.6-2}, we will use a homework result showing that 
\begin{equation}
\|A-U_CY\|_F^2 = \|(I-U_CU_C^T)A\|_F^2 + \|U_C^TA-Y\|_F^2	\quad \forall Y\text{ with }\operatorname{rk}(Y)\leq k.
\end{equation}
Now the minimum on the right hand side (and hence also on the left hand side) under the constraint $\operatorname{rank} \leq k$ is archived at $y=(U_C^TA)_k$, which shows \eqref{lem:3.6-2}.
\end{proof}
\end{lemma}
\mymarginpar{\tiny{6. Lecture \\ 06.12.2022}} 
\end{theorem}
\end{document}
