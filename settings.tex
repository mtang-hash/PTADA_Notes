\documentclass[12pt,a4paper]{article}
\usepackage[english]{babel}
\usepackage[T1]{fontenc}
\usepackage[utf8]{inputenc}
% \usepackage{lmodern}
\usepackage{lmodern}
\usepackage[left=28mm,top=28mm,right=28mm,bottom=28mm] {geometry}
\usepackage{amsfonts}

\usepackage{mathrsfs}
\usepackage[intlimits]{amsmath}
\usepackage{mathtools}

\usepackage{mathabx}
\usepackage{bbm}
\usepackage{stmaryrd}
\usepackage{relsize}
\usepackage{etoolbox}
\usepackage{lscape}
\usepackage{afterpage}
\usepackage{makecell}
\usepackage[shortlabels]{enumitem}
\usepackage{amssymb}
\usepackage{stmaryrd}
\usepackage{cancel}
\usepackage{mdframed}
\usepackage{framed}{}
\usepackage{tablefootnote} 
\usepackage{listings}
\usepackage{amsthm}
%\usepackage[dvipsnames]{xcolor}
\usepackage[x11names,dvipsnames]{xcolor}
\colorlet{myGray}{gray!3!white}
\usepackage{etoolbox}
\usepackage[all]{xy}
\usepackage{tikz}
%\usepackage{subfigure}
%\usepackage{bbold}
\usetikzlibrary{calc}
\usetikzlibrary{shapes.geometric}
\usetikzlibrary{arrows.meta}
\usetikzlibrary{decorations.pathreplacing}
\usetikzlibrary{fadings}
\usepackage{thmtools}
\usepackage{stmaryrd}
\usepackage[ruled,vlined]{algorithm2e}
%\usepackage{stix}
%\let\Hermaphrodite\relax
\usepackage[font=small]{caption}
\usepackage{subcaption}
\usepackage[final]{pdfpages}
\usepackage{hhline}
\let\Sun\relax
\let\Moon\relax
\let\Mercury\relax
\let\Venus\relax
\let\Earth\relax
\let\Mars\relax
\let\Jupiter\relax
\let\Saturn\relax
\let\Uranus\relax
\let\Neptune\relax
\let\Pluto\relax
\let\Gemini\relax
\let\Leo\relax
\let\Libra\relax
\let\Scorpio\relax
\let\Aries\relax
\let\Taurus\relax
\usepackage{marvosym}
\usepackage[
   pdfpagelabels=true,
   pdftitle={Probabilistic Techniques and Algorithms in Data Analysis},
   pdfauthor={Min Tang},
   hidelinks
 ]{hyperref}
\usepackage{esint}
\usepackage{bookmark}
\usepackage[usenames,dvipsnames]{pstricks}
\usepackage{epsfig}
\usepackage{pst-grad} % For gradients
\usepackage{pst-plot} % For axes
\usepackage[space]{grffile} % For spaces in paths
\usepackage{etoolbox} % For spaces in paths
\usepackage{listings}
\usepackage[style=numeric,sorting=none,backend=bibtex]{biblatex}
\usepackage{dsfont}
\addbibresource{literature.bib}
\lstset{
  frame=none,
  xleftmargin=0pt,
  %stepnumber=1,
  %numbers=left,
  %numbersep=5pt,
  backgroundcolor = \color{myGray},
  numberstyle=\ttfamily\tiny\color[gray]{0.3},
  belowcaptionskip=\bigskipamount,
  captionpos=b,
  escapeinside={*'}{'*},
  language=haskell,
  tabsize=2,
  emphstyle={\bf},
  commentstyle=\it,
  stringstyle=\mdseries\rmfamily,
  showspaces=false,
  keywordstyle=\bfseries\rmfamily,
  columns=flexible,
  basicstyle=\small\sffamily,
  showstringspaces=false,
  morecomment=[l]\%,
}
\makeatletter % For spaces in paths
\patchcmd\Gread@eps{\@inputcheck#1 }{\@inputcheck"#1"\relax}{}{}
\makeatother

\usetikzlibrary{cd}
\usetikzlibrary{calc}
\def\checkmark{\tikz\fill[scale=0.4](0,.35) -- (.25,0) -- (1,.7) -- (.25,.15) -- cycle;} 
\theoremstyle{definition}
\newtheorem{definition}{Definition}
\newtheorem{problem}[definition]{Problem}
\theoremstyle{theorem}
\newtheorem{theorem}[definition]{Theorem}
\newtheorem*{theorem*}{Theorem}
\newtheorem{prop}[definition]{Proposition}
\newtheorem{lemma}[definition]{Lemma}
\newtheorem{corollary}[definition]{Corollary}
\newtheorem{example}[definition]{Example}
\newtheorem*{example*}{Example}
\theoremstyle{definition}
\newtheorem{remark}[definition]{Remark}
\newtheorem*{remark*}{Remark}
\AfterEndEnvironment{lemma}{\noindent\ignorespaces}
\AfterEndEnvironment{definition}{\noindent\ignorespaces}
\AfterEndEnvironment{example}{\noindent\ignorespaces}
\AfterEndEnvironment{theorem}{\noindent\ignorespaces}
% \AfterEndEnvironment{satz}{\noindent\ignorespaces}
% \AfterEndEnvironment{korollar}{\noindent\ignorespaces}
\AfterEndEnvironment{remark}{\noindent\ignorespaces}
\AfterEndEnvironment{remark*}{\noindent\ignorespaces}
% \AfterEndEnvironment{remark'}{\noindent\ignorespaces}
% \AfterEndEnvironment{proposition}{\noindent\ignorespaces}
% \AfterEndEnvironment{proof}{\noindent\ignorespaces}
\let\existstemp\exists
\let\foralltemp\forall
\newcommand{\tikzmark}[1]{\tikz[overlay,remember picture] \node (#1) {};}
\newcommand{\vsubset}{\rotatebox[origin=c]{90}{$\subset$}}
\newcommand{\vphi}{\phi}
\newcommand{\ol}{\overline}
%Differentiation
\newcommand{\D}{\, \mathrm{d} }
%Bold Symbols
\newcommand{\E}{\mathbb{E}}

\newcommand{\R}{\mathbb{R}}
\newcommand{\C}{\mathbb{C}}
\newcommand{\N}{\mathbb{N}}
\newcommand{\Q}{\mathbb{Q}}
%Calligraphic Symbols
\newcommand{\ZZ}{\mathcal{Z}}
\newcommand{\II}{\mathcal{I}}
\newcommand{\FF}{\mathcal{F}}
\newcommand{\QQ}{\mathcal{Q}}
\newcommand{\EE}{\mathcal{E}}
\newcommand{\PP}{\mathcal{P}}
\newcommand{\TT}{\mathcal{T}}
\newcommand{\MM}{\mathcal{M}}
\newcommand{\HH}{\mathscr{H}}
\newcommand{\RR}{\mathscr{R}}
\newcommand{\BB}{\mathscr{B}}
\newcommand{\CC}{\mathscr{C}}
\newcommand{\DD}{\mathscr{D}}
\newcommand{\LL}{\mathscr{L}}
\newcommand{\NN}{\mathscr{N}}
\renewcommand{\AA}{\mathscr{A}}
% Operator names
\newcommand{\id}{\operatorname{id}}
\newcommand{\Hom}{\operatorname{Hom}}
\newcommand{\del}{\partial}
\newcommand{\GL}{\operatorname{GL}}
\newcommand{\vol}{\operatorname{vol}}
\newcommand{\Var}{\operatorname{Var}} 
\newcommand{\Cov}{\operatorname{Cov}}
\newcommand{\End}{\operatorname{End}}
\newcommand{\SL}{\operatorname{SL}}
\newcommand{\Aff}{\operatorname{Aff}}
\newcommand{\Isom}{\operatorname{Isom}}
\newcommand{\Trans}{\operatorname{Trans}}
\newcommand{\Bild}{\begin{tiny}(Bild hier)\end{tiny}}
\newcommand{\Ueb}{\begin{tiny}\textbf{(Ü)}\end{tiny}}
\newcommand{\te}{\text}
\newcommand{\comment}[1]{}
\newcommand{\drawaline}{\rule{\textwidth}{0.2pt}}
%\renewcommand{\def}{\definition}
\renewcommand{\ker}{\operatorname{ker}}
\renewcommand*{\forall}{\foralltemp\mkern2mu}
\renewcommand{\emptyset}{\varnothing}
\renewcommand{\Re}{\operatorname{Re}}
\renewcommand{\O}{\operatorname{O}}
\renewcommand{\Im}{\operatorname{im}}
\renewcommand{\qedsymbol}{$\blacksquare$}
\renewcommand{\phi}{\varphi}
\makeatletter 
\AfterEndEnvironment{mdframed}{%
 \tfn@tablefootnoteprintout% 
 \gdef\tfn@fnt{0}% 
}
\numberwithin{equation}{section}
\numberwithin{definition}{section}
\numberwithin{theorem}{section}
\numberwithin{example}{section}
\numberwithin{lemma}{section}
\numberwithin{prop}{section}
\numberwithin{corollary}{section}
\numberwithin{problem}{section}
\numberwithin{remark}{section}
\usepackage{xpatch}

% \makeatletter
% \AtBeginDocument{\xpatchcmd{\@thm}{\thm@headpunct{.}}{\thm@headpunct{}}{}{}}
% \makeatother
\newcommand{\attvis}[2]{\definecolor{att}{rgb}{1, #2, #2} \colorbox{att}{#1}}


\setlist[enumerate,1]{label={(\roman*)}}
\DeclareUnicodeCharacter{2212}{-}
\DeclareRobustCommand{\mymarginpar}[1]{%
 \marginpar[\raggedleft#1]{\raggedright#1}}
\pagestyle{headings}
% \linespread{1.1}
